%%=============================================================================
%% Samenvatting
%%=============================================================================

% TODO: De "abstract" of samenvatting is een kernachtige (~ 1 blz. voor een
% thesis) synthese van het document.
%
% Een goede abstract biedt een kernachtig antwoord op volgende vragen:
%
% 1. Waarover gaat de bachelorproef?
% 2. Waarom heb je er over geschreven?
% 3. Hoe heb je het onderzoek uitgevoerd?
% 4. Wat waren de resultaten? Wat blijkt uit je onderzoek?
% 5. Wat betekenen je resultaten? Wat is de relevantie voor het werkveld?
%
% Daarom bestaat een abstract uit volgende componenten:
%
% - inleiding + kaderen thema
% - probleemstelling
% - (centrale) onderzoeksvraag
% - onderzoeksdoelstelling
% - methodologie
% - resultaten (beperk tot de belangrijkste, relevant voor de onderzoeksvraag)
% - conclusies, aanbevelingen, beperkingen
%
% LET OP! Een samenvatting is GEEN voorwoord!

%%---------- Nederlandse samenvatting -----------------------------------------
%
% TODO: Als je je bachelorproef in het Engels schrijft, moet je eerst een
% Nederlandse samenvatting invoegen. Haal daarvoor onderstaande code uit
% commentaar.
% Wie zijn bachelorproef in het Nederlands schrijft, kan dit negeren, de inhoud
% wordt niet in het document ingevoegd.

\IfLanguageName{english}{%
\selectlanguage{dutch}
\chapter*{Samenvatting}
\lipsum[1-4]
\selectlanguage{english}
}{}

%%---------- Samenvatting -----------------------------------------------------
% De samenvatting in de hoofdtaal van het document

\chapter*{\IfLanguageName{dutch}{Samenvatting}{Abstract}}

Door de omschakeling naar hernieuwbare energie wordt volop ingezet op elektrificatie. Dit vraagt enerzijds een modernisering van het distributienetwerk voor elektriciteit, maar ook de inzet van slimme technologieën om dit distributienetwerk te ontlasten. Een voorbeeld hiervan is de digitale elektriciteitsmeter die met behulp van allerhande apps 'slim' kan gemaakt worden en gezinnen zo meer inzicht kan geven in hun elektriciteitsverbruik. \\

Dit onderzoek zal trachten te achterhalen hoe een app die een digitale meter aanstuurt kan geoptimaliseerd worden met behulp van artificiële intelligentie (AI) . De meeste van de  apps die momenteel bestaan geven immers enkel op een passieve manier weer wat het huidige verbuik en de huidige productie is.  Er wordt van de verbruiker verwacht dat hij of zij op basis van deze inzichten de nodige acties onderneemt om het elektriciteitsverbruik bij te sturen. De Vlaamse Regulator van de Elektriciteits- en Gasmarkt (VREG) (2021) heeft vastgesteld dat het slim maken van digitale elektriciteitsmeters voorlopig niet tot de gewenste gedragsverandering en spreiding van het verbruik van elektriciteit leidt. \\

In het eerste deel van het onderzoek zal worden nagegaan welke apps er reeds ontwikkeld werden en wat hun tekortkomingen zijn. Het verdere onderzoek zal vervolgnes een app gaan ontwikkelen die met behulp van artificiële intelligentie en een weer API het elektriciteitsverbruik automatisch en dus actief zal bijsturen en spreiden. Meer concreet zal AI worden gebruikt om voorspellingen te doen op basis van historische verbruiksdata die dan gecombineerd worden met weersvoorspellingen die verkregen worden via een weer API. De app zal dan op basis van de gecombineerde voorspellingen een warmtepomp en sanitaire toestellen automatisch gaan in- of uitschakelen. Daarbij zal de gebruiker wel nog steeds de mogelijkheid geboden worden om manueel tussen te komen. \\

Het verwachte resultaat is dat het elektriciteitsverbruik op die manier wel meer gestuurd en gespreid zal worden. Dit zal niet alleen tot een ontlasting van het distributienetwerk voor elektriciteit leiden, maar zal ook de energiekosten voor de gezinnen drukken.
