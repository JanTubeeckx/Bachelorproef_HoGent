%%=============================================================================
%% Voorwoord
%%=============================================================================

\chapter*{\IfLanguageName{dutch}{Woord vooraf}{Preface}}%
\label{ch:voorwoord}

%% TODO:
%% Het voorwoord is het enige deel van de bachelorproef waar je vanuit je
%% eigen standpunt (``ik-vorm'') mag schrijven. Je kan hier bv. motiveren
%% waarom jij het onderwerp wil bespreken.
%% Vergeet ook niet te bedanken wie je geholpen/gesteund/... heeft

Door mijn huidige opdracht als consultant bij de distributienetbeheerder Fluvius kwam ik in aanraking met de uitdagingen waarmee de energiemarkt momenteel geconfronteerd wordt. Door het toenemend belang van alternatieve energiebronnen om de CO2 uitstoot terug te dringen, is onze samenleving in sneltempo aan het elektrificeren. Dit zorgt voor een enorme extra belasting van het bestaande distributienetwerk voor elektriciteit. Deze toenemende belasting kan onmogelijk worden opgevangen door een uitbreiding van dit distributienetwerk, wat bovendien te grote investeringen zou vereisen. Daarom wordt volop gekeken naar allerlei manieren om het elektriciteitsnetwerk optimaler te gaan gebruiken en zo te ontlasten. Een van die manieren is een betere spreiding van het elektriciteitsverbruik door gezinnen. Daarvoor bestaan er al heel wat apps, maar de meeste daarvan vergen een manuele tussenkomt van de gebruiker. Uit onderzoek van de Vlaamse Regulator van de Elektriciteits- en Gasmarkt (VREG) is gebleken dat dit niet tot een gedragsverandering en dus een betere spreiding van het elektriciteitsverbruik leidt. Dit gaf me de inspiratie om via deze bachelorproef te achterhalen of een app met behulp van artificiële intelligentie voor een automatische bijsturing en spreiding van het elektriciteitsverbruik kan zorgen. \\

In de eerste plaats wil ik graag mijn vriendin Thelma bedanken die niet enkel tijdens deze bachelorproef, maar al vijf jaar lang mijn vaste steun en toeverlaat tijdens mijn studies is geweest. Daarnaast ook een dankwoord aan Thomas Tomme die mij als co-promoter en energiespecialist waardevolle inzichten heeft verschaft over de huidige energie- en flexmarkt. Verder wil ik ook mijn promotor Sebastiaan Labijn bedanken voor zijn begeleiding en feedback. Ook mijn schoonouders verdienen een dankwoord voor de terbeschikkingstelling van hun woning als testomgeving. Een laaste bedanking gaat tenslotte naar Melvin, de trouwe viervoeter die de eenzame studiemomenten opvulde met zijn immer opgewekte aanwezigheid.