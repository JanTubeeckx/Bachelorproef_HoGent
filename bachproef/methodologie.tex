%%=============================================================================
%% Methodologie
%%=============================================================================

\chapter{\IfLanguageName{dutch}{Methodologie}{Methodology}}%
\label{ch:methodologie}

%% TODO: In dit hoofstuk geef je een korte toelichting over hoe je te werk bent
%% gegaan. Verdeel je onderzoek in grote fasen, en licht in elke fase toe wat
%% de doelstelling was, welke deliverables daar uit gekomen zijn, en welke
%% onderzoeksmethoden je daarbij toegepast hebt. Verantwoord waarom je
%% op deze manier te werk gegaan bent.
%% 
%% Voorbeelden van zulke fasen zijn: literatuurstudie, opstellen van een
%% requirements-analyse, opstellen long-list (bij vergelijkende studie),
%% selectie van geschikte tools (bij vergelijkende studie, "short-list"),
%% opzetten testopstelling/PoC, uitvoeren testen en verzamelen
%% van resultaten, analyse van resultaten, ...
%%
%% !!!!! LET OP !!!!!
%%
%% Het is uitdrukkelijk NIET de bedoeling dat je het grootste deel van de corpus
%% van je bachelorproef in dit hoofstuk verwerkt! Dit hoofdstuk is eerder een
%% kort overzicht van je plan van aanpak.
%%
%% Maak voor elke fase (behalve het literatuuronderzoek) een NIEUW HOOFDSTUK aan
%% en geef het een gepaste titel.

Supervised machine learning en meer specifiek regressie. (meest gebruikte machine learning techniek) \\

Multiple linear regression? Model dat gebruikt wordt wanneer meer dan 2 variabelen: historische productie + weersvoorspelling weer API \\

To do: \\
- Selecteer de beste ML-technieken om voorspellingen te maken \\
- Bepaal de accuraatheid van het regressiemodel (zie colab HoGent + rapport PXL) \\
- Kies model met beste score \\

Eerst zal de verzamelde data moeten gecleaned, getransformeerd en geanalyseerd worden. \\

Ensemble regressors en meer bepaald gradient boosting lijken het beste model te zijn want ANMAE (Absolute Normalized Mean Absolute Error) en PRMSE (??) laagste \autocite{Khasawneh2024} en \autocite{Tercha2024}. \\

Korte termijn voorspelling elektriciteitsverbruik: ook daar eXtreme Gradient Boosting het beste ML algoritme \autocite{Irankhah2024}. \\

Vaatwas en wasmachine verbruiken gemiddeld 1000 W per uur dus vanaf dan mogen deze toestellen ingeschakeld worden. Toestellen ingeven in app en daarachter zit bepaald wattage en vanf dat wattage zal het ingeschakeld worden.


