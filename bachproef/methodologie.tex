%%=============================================================================
%% Methodologie
%%=============================================================================

\chapter{\IfLanguageName{dutch}{Methodologie}{Methodology}}%
\label{ch:methodologie}

%% TODO: In dit hoofstuk geef je een korte toelichting over hoe je te werk bent
%% gegaan. Verdeel je onderzoek in grote fasen, en licht in elke fase toe wat
%% de doelstelling was, welke deliverables daar uit gekomen zijn, en welke
%% onderzoeksmethoden je daarbij toegepast hebt. Verantwoord waarom je
%% op deze manier te werk gegaan bent.
%% 
%% Voorbeelden van zulke fasen zijn: literatuurstudie, opstellen van een
%% requirements-analyse, opstellen long-list (bij vergelijkende studie),
%% selectie van geschikte tools (bij vergelijkende studie, "short-list"),
%% opzetten testopstelling/PoC, uitvoeren testen en verzamelen
%% van resultaten, analyse van resultaten, ...
%%
%% !!!!! LET OP !!!!!
%%
%% Het is uitdrukkelijk NIET de bedoeling dat je het grootste deel van de corpus
%% van je bachelorproef in dit hoofstuk verwerkt! Dit hoofdstuk is eerder een
%% kort overzicht van je plan van aanpak.
%%
%% Maak voor elke fase (behalve het literatuuronderzoek) een NIEUW HOOFDSTUK aan
%% en geef het een gepaste titel.

Het onderzoek van deze bachelorproef bestaat uit twee delen. In het eerste deel wordt een literatuurstudie gemaakt om de huidige stand van zaken met betrekking tot het onderwerp van deze bachelorproef inzichtelijk te maken. Er wordt gestart met een overzicht van de functionaliteiten en tekortkomingen van de apps die er momenteel bestaan om een digitale elektriciteitsmeter slimmer te maken. Daarna wordt toegelicht hoe met behulp van artificiële intelligentie (AI) en meer specifiek machine learning (ML) voorspellingen kunnen gemaakt worden. Er wordt daarbij kort ingegaan op de ML algoritmes die het vaakst gebruikt worden om zonnestraling en stroomproductie van zonnepanelen te voorspellen en hoe de prestaties van deze algoritmes kunnen beoordeeld worden. Vervolgens wordt een overzicht gegeven van de onderzoeken die reeds gevoerd zijn naar de voorspelling van zonnestraling en/of stroomproductie van zonnepanelen met behulp van machine learning. Daarbij wordt ook uitgelegd welke data hiervoor zal gebruikt worden in deze bachelorproef. De literatuurstudie wordt afgesloten met een toelichting over de aansturing van slimme apparaten en de problematiek van het ontbreken van een gestandardiseerd protocol hiervoor. \\

In het tweede deel van deze bachelorproef wordt de ontwikkeling van een proof of concept beschreven. Er zal een mobiele app ontwikkeld worden die in de eerste plaats de elektriciteitsconsumptie en -productie en de voorspelling van de stroomproductie van de zonnepanelen inzichtelijk maakt. Deze app zal vervolgens automatisch een slimme stekker gaan aansturen op basis van de voorspelling die via de toepassing van machine learning gemaakt werd.


