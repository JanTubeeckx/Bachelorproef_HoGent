%%=============================================================================
%% Conclusie
%%=============================================================================

\chapter{\IfLanguageName{dutch}{Conclusie}{Conclusion}}%
\label{ch:conclusie}

% TODO: Trek een duidelijke conclusie, in de vorm van een antwoord op de
% onderzoeksvra(a)g(en). Wat was jouw bijdrage aan het onderzoeksdomein en
% hoe biedt dit meerwaarde aan het vakgebied/doelgroep? 
% Reflecteer kritisch over het resultaat. In Engelse teksten wordt deze sectie
% ``Discussion'' genoemd. Had je deze uitkomst verwacht? Zijn er zaken die nog
% niet duidelijk zijn?
% Heeft het onderzoek geleid tot nieuwe vragen die uitnodigen tot verder 
%onderzoek?
Met deze bachelorproef werd getracht een antwoord te formuleren op de vraag of artificiële intelligentie kan worden aangewend om het elektriciteitsverbruik van gezinnen automatisch bij te sturen en zo de elektriciteitskosten te drukken. De behoefte aan dit onderzoek kwam er na de vaststelling dat heel wat gezinnen hun elektriciteitsconsumptie niet aanpassen wanneer ze over een app beschikken die hun energieverbruik en -productie enkel inzichtelijk maakt. Blijkbaar zorgt dit niet voor een gedragswijziging en was het dus nuttig om na te gaan hoe een app dan zelf het elektriciteitsverbruik kan bijsturen. Met andere woorden, kan een app door middel van artificiële intelligentie zo 'slim' gemaakt worden dat deze zelf apparaten in een gezinswoning kan in- of uitschakelen om zo het elektricteitsverbruik af te stemmen op de elektriciteitsproductie van de zonnepanelen? Het idee was om de stroomproductie van een PV-installatie te gaan voorspellen met behulp van machine learning om dan vervolgens slimme toestellen of slimme stekkers te gaan aansturen op basis van de gemaakt voorspelling. Dit zou worden uitgewerkt in zelf ontwikkelde app die als proof of concept zou dienen.\\

Om deze vraag te kunnen beantwoorden was vooreerst de juiste testomgeving noodzakelijk. Deze moest bestaan uit een woning met een digitale elektriciteitsmeter en een PV-installatie (zonnepanelen). Vervolgens moest werden nagegaan op welke manier een digitale elektriciteitsmeter en de omvormer van de PV-installatie op continue wijze konden worden uitgelezen. De oplossing hiervoor werd gevonden in de Raspberry Pi minicomputer. Dit toestel kon via wifi worden aangesloten op het thuisnetwerk van de woning die als testomgeving dienst deed. Vervolgens werden hierop Python scripts uitgevoerd die de gegevens van de elektriciteitsmeter en de omvormer van de zonnepanelen uitlazen en wegschreven naar een Influx databank. Gedurende de testperiode bleek deze oplossing zeer robuust en stabiel. Enige nadeel was dat wanneer het signaal van het wifinetwerk wegviel door een stroomonderbreking, het uitlezen gestopt werd. Dit werd uiteindelijk opgevangen door de Python scripts te voorzien van een automatisch heropstart commando. \\

Na het opzetten van de nodige hardware en het uitschrijven van de Python-scripts, werd onderzocht welke vorm van artificiële intelligentie kon worden aangewend om de voorspellingen van de stroomproductie te gaan maken. Na een uitgebreide literatuurstudie werd geopteerd voor de machine learning techniek van extreme gradient boosting (XGBoost). Deze methode is specifiek ontworpen om grootschalige datasets efficiënt te verwerken en leidt tot nauwkeurige voorspellingen. De toepassing van dit algoritme op de historische zonnestralingsdata van de afgelopen vijf jaar die via de CAMS Radiation Service (CRS) van de Copernicus Atmosphere Monitoring Service (\href{https://atmosphere.copernicus.eu}{CAMS}) verkregen werd, leidde inderdaad tot correcte voorspellingen. Om de gemaakte voorspelling nog nauwkeuriger te maken werd de historische zonnestralingsdata in de eerste plaats gecombineerd met historische weerdata. Uit onderzoek blijkt immers dat de temperatuur en luchtvochtigheid een invloed hebben op de elektricteitsproductie van zonnepanelen. Vervolgens werd de bekomen voorspelling gecombineerd met de weersvoorspelling voor de voorspelde periode. Zo kon dan een voldoende accurate voorspelling van de stroomproductie gemaakt worden. Om de uitgelezen elektricteitsgegevens en de voorspelling van de elektriciteitsproductie te visualiseren werd tenslotte een iOS app ontwikkeld. \\

De laatste stap van het onderzoek zou er dan in bestaan om de gemaakte voorspelling van de stroomproductie te gebruiken om (sanitaire) toestellen aan te sturen. Bij het formuleren van het onderzoeksvoorstel van deze bachelorproef werd ervan uitgegaan dat er een testomgeving beschikbaar zou zijn met zonnepanelen en een recente warmtepomp waarmee ook sanitair water wordt opgewarmd. Het is namelijk zo dat de huidige warmtepompen kunnen aangestuurd worden via een HEMS (home energy management system) en bijgevolg de ideale toestellen vormen om aan te sturen, zeker als ze ook gebruikt worden voor de opwarming van sanitair water. Zo kan de stroom van de zonnepanelen overdag immers gebruikt worden om het water in het boilervat van de warmtepomp op te warmen. De testomgeving die voor dit onderzoek gebruikt werd, beschikte echter niet over een warmtepomp. En al gauw bleek hier het grootste obstakel voor dit onderzoek. Andere toestellen in een gezinswoning die interessant zijn om via een HEMS aan te sturen, zijn een wasmachine en vaatwasser. Deze worden meestal op dagelijkse basis gebruikt en zijn tevens toestellen die redelijk wat energie verbruiken. Probleem is evenwel dat in de meeste gezinswoningen en ook in de woning van de testomgeving, deze toestellen niet 'slim' zijn en dus niet via wifi en een app kunnen worden aangestuurd. Als oplossing werd daarom teruggegrepen naar slimme stekkers. Daar stelde zich evenwel opnieuw een probleem. De huidige wasmachines en vaatwassers hebben meestal een elektronische startknop die enkel bediend kan worden als het toestel van elektriciteit voorzien wordt. Het toestel kan dus niet op zodanig wijze klaar gezet worden, dat het een wasprogramma start wanneer het stroom krijgt. De gebruiker zal nog steeds moeten tussenkomen om de startknop van het apparaat in te duwen. Hoewel dit voor het onderzoek van deze bachelorproef een probleem vormde, is de verwachting dat in de toekomst sanitaire toestellen standaard uitgerust zullen zijn met een wifiverbinding en dus via een app kunnen aangestuurd worden. Aandachtspunt daarbij zal zijn dat het op dit moment nog niet duidelijk is of daarbij een standaard communicatieprotocol zal gehanteerd worden, of dat elke fabrikant een eigen protocol blijft gebruiken. In het laatste geval zal de aansturing met een HEMPS-app een stuk moeilijker zijn omdat de app dan met al die verschillende communicatieprotocollen zal moeten kunnen omgaan. Dit maakt de ontwikkeling en het onderhoud ervan complexer. \\

Als alternatief werd in dit onderzoek gekozen voor de batterijlader van een elektrische fiets. Deze lader kon zonder problemen ingeschakeld worden met behulp van een slimmer stekker. De lader kon elke avond worden aangesloten en startte de volgende dag met opladen van zodra de zonnepalen een vermogen van 800 Watt produceerden. Deze grens van 800 Watt werd vastgelegd na een testperiode van 3 weken. Uit de vaststellingen bleek immers dat op deze manier gemiddeld het grootste aantal uren aan stroomproductie van een gemiddelde dag kon bekomen worden.  Er werd gedurende drie weken getest met deze opstelling en daaruit bleek dat de ontwikkelde app in staat was om de slimme stekker en dus de batterijlader op het correcte moment in te schakelen. Wat het exacte bedrag van de uitgespaarde elektriciteitskost is, werd niet berekend. Dat er evenwel een elektriciteitskost werd uitgespaard, is wel duidelijk. De fietsbatterij werd steevast 's avonds opgeladen, wanneer er geen zonne-energie meer voorhanden was. De kost van dit elektriciteitsverbruik zal dus voortaan uitgespaard worden. Verder onderzoek zal moeten uitwijzen hoe groot de kostenbesparing van elektriciteit op jaarbasis kan zijn, indien meerdere apparaten door de ontwikkelde app worden aangestuurd. \\