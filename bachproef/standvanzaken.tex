\chapter{\IfLanguageName{dutch}{Stand van zaken}{State of the art}}%
\label{ch:stand-van-zaken}

% Tip: Begin elk hoofdstuk met een paragraaf inleiding die beschrijft hoe
% dit hoofdstuk past binnen het geheel van de bachelorproef. Geef in het
% bijzonder aan wat de link is met het vorige en volgende hoofdstuk.

% Pas na deze inleidende paragraaf komt de eerste sectiehoofding.

%Dit hoofdstuk bevat je literatuurstudie. De inhoud gaat verder op de inleiding, maar zal het onderwerp van de bachelorproef *diepgaand* uitspitten. De bedoeling is dat de lezer na lezing van dit hoofdstuk helemaal op de hoogte is van de huidige stand van zaken (state-of-the-art) in het onderzoeksdomein. Iemand die niet vertrouwd is met het onderwerp, weet nu voldoende om de rest van het verhaal te kunnen volgen, zonder dat die er nog andere informatie moet over opzoeken \autocite{Pollefliet2011}.

%Je verwijst bij elke bewering die je doet, vakterm die je introduceert, enz.\ naar je bronnen. In \LaTeX{} kan dat met het commando \texttt{$\backslash${textcite\{\}}} of \texttt{$\backslash${autocite\{\}}}. Als argument van het commando geef je de ``sleutel'' van een ``record'' in een bibliografische databank in het Bib\LaTeX{}-formaat (een tekstbestand). Als je expliciet naar de auteur verwijst in de zin (narratieve referentie), gebruik je \texttt{$\backslash${}textcite\{\}}. Soms is de auteursnaam niet expliciet een onderdeel van de zin, dan gebruik je \texttt{$\backslash${}autocite\{\}} (referentie tussen haakjes). Dit gebruik je bv.~bij een citaat, of om in het bijschrift van een overgenomen afbeelding, broncode, tabel, enz. te verwijzen naar de bron. In de volgende paragraaf een voorbeeld van elk.

%\textcite{Knuth1998} schreef een van de standaardwerken over sorteer- en zoekalgoritmen. Experten zijn het erover eens dat cloud computing een interessante opportuniteit vormen, zowel voor gebruikers als voor dienstverleners op vlak van informatietechnologie~\autocite{Creeger2009}.

%Let er ook op: het \texttt{cite}-commando voor de punt, dus binnen de zin. Je verwijst meteen naar een bron in de eerste zin die erop gebaseerd is, dus niet pas op het einde van een paragraaf. \\

Met een digitale elektriciteitsmeter kunnen gezinnen hun elektriciteitsverbruik makkelijk opvolgen. Dat kan in de eerste plaats gratis via het online energieportaal \href{https://login.fluvius.be/klanten.onmicrosoft.com/b2c_1a_customer_signup_signin/oauth2/v2.0/authorize?client_id=91bb9a0a-f45d-491a-ae0b-43324fbc343a&scope=openid%20profile%20offline_access&redirect_uri=https%3A%2F%2Fmijn.fluvius.be%2Fredirect&client-request-id=90c12c72-7d7b-428b-98fc-5d7956e53a60&response_mode=fragment&response_type=code&x-client-SKU=msal.js.browser&x-client-VER=2.23.0&client_info=1&code_challenge=jz-1E8AwB15UEa352eC_5x6zDtAtwp3Je6jrFVdGKjk&code_challenge_method=S256&nonce=cee3d720-d931-4b13-b0b5-c473169ca6fd&state=eyJpZCI6IjRhM2I3M2NkLTgyZjgtNDFjOC05NzAyLTEwMTNjNjNkNjNhMyIsIm1ldGEiOnsiaW50ZXJhY3Rpb25UeXBlIjoicmVkaXJlY3QifX0%3D}{Mijn Fluvius}. Daarnaast bestaan er ook heel wat gratis of betalende online apps die met een digitale meter kunnen verbonden worden om elektriciteitsverbruik op te volgen en eventueel bij te sturen. Zo’n slimme toepassingen heten in het jargon ‘HEMS’ (Home Energy Management System) of ‘CEMS’ (Customer Energy Management System). De aansluiting van deze apps gebeurt via de gebruikerspoorten (P1 en S1) van de digitale elektriciteitsmeter. Beide gebruikerspoorten zijn complementair en geschikt voor verschillende toepassingen. De P1-poort stuurt de elektriciteitsdata per seconde uit. Via de ‘snelle’ S1-poort worden ruwe data aan een zeer hoge frequentie ter beschikking gesteld aan een app of slimme thermostaat. Dit laat gedetailleerde verbruikersfeedback en sturing toe. Recente digitale elektriciteitsmeters hebben evenwel geen S1-poort meer. De verbinding tussen het meettoestel en een app gebeurt in de meeste gevallen via een wifiverbinding of 4G. Een overzicht van deze toepassingen vindt men op de website \href{https://maakjemeterslim.be/}{www.maakjemeterslim.be}.

\section{\IfLanguageName{dutch}{Overzicht functionaliteiten bestaande apps}{Overview functionalities existing apps}}%
\label{sec:overzicht functionaliteiten bestaande apps}

De bestaande apps bieden allemaal de mogelijkheid om elektriciteitsverbruik in real time op te volgen. Zo kan de gebruiker per dag, per week of per maand nagaan hoeveel elektriciteit er verbruikt en/of geproduceerd werd. Het meten van de energiekosten per huishoudtoestel is ook meestal standaard voorzien, maar geldt even vaak als een optie. De meeste apps sporen ook sluipverbruik op en bieden gepersonaliseerde tips voor energiebesparing. Sommige apps bieden tenslotte ook de mogelijkheid om het gemeten elektriciteitsverbruik naast weerdata te leggen en op die manier na te gaan hoeveel stroom men verbruikt bij bepaalde weersomstandigheden \autocite{Deman2021}. Toen ik vorig jaar mijn onderzoeksvoorstel voor deze bachelorproef formuleerde en uitschreef voor het vak 'Research methods' was er nog geen enkele app die de inzichten in het elektriciteitsverbruik gebruikte om automatisch apparten aan te sturen. Het was toen aan de gebruiker zelf om op basis van de gevevens die de app hem toonde actie te ondernemen en toestellen te gaan in- of uitschakelen. Zoals reeds eerder werd vermeld, leidde dit niet altijd tot de gewenste gedragsverandering \autocite{Wemyss2019}, \autocite{Mack2019} en  \autocite{VREG2021}. Ondertussen zijn een aantal apps daarom uitgebreid met de optie om (huishoud)apparaten automatisch in te schakelen via slimme stekkers. Zes apps voorzien momenteel in de optie om toestellen in te schakelen volgens elektriciteitstarieven of het overschot aan zelf geproduceerde stroom van zonnepanelen, daarbij telkens rekening houdend met het piekverbruik zodat een hoger capaciteitstarief vermeden wordt. Dit gebeurt met bijgeleverde slimme stekkers die via wifi kunnen aangestuurd worden. De automatische aansturing gebeurt echter steeds op het moment zelf, wanneer de app merkt dat er een overschot aan elektricteitsproductie is of de elektriciteitsprijzen laag zijn. Tot op heden is er evenwel nog geen app die gebruik maakt van artificiële intelligentie om de zelf geproduceerde zonne-energie te voorspellen. Dit laat de gebruiker immers toe om de vaatwasser of wasmachine op voorhand  in te laden, zodat de automatische aansturing ervan de volgende dag ook effectief kan gebeuren. \\

\section{\IfLanguageName{dutch}{Voorspellingen maken met AI}{Making predictions with AI}}%
\label{sec:voorspellingen maken met AI}

Artificiële intelligentie (AI) verwijst naar de mogelijkheid van applicaties of machines om menselijke vaardigheden te tonen, zoals redeneren, leren en plannen. De term AI wordt vaak gebruikt om andere gerelateerde technieken aan te duiden, zoals machine learning (ML) en deep learning. ML en deeplearning kunnen inderdaad onder de verzamelnaam AI vallen, maar omgekeerd is dit niet altijd het geval. De verschillende technieken van AI hebben immers elk hun eigen doel en opzet. Zo is ML bijvoorbeeld gericht op de studie en de ontwikkeling van algoritmen die kunnen leren of hun prestaties kunnen verbeteren op basis van de data waarmee ze gevoed worden. Machine learning is daarom zeer geschikt om voorspellingen te maken op basis van data uit het verleden. \\

\subsection{\IfLanguageName{dutch}{Machine learning}{Machine learning}}%
\label{sec:machine learning}

Machine learning kan opnieuw onderverdeeld worden in drie verschillende categorieën, namelijk supervised learning, unsupervised learning en reinforcement learning. 

\subsubsection{Supervised learning}

Bij supervised machine learning wordt een algoritme door de mens aangeleerd welke conclusies het moet trekken. Er wordt daarbij gewerkt met gelabelde data, wat betekent dat voor elke invoer de gewenste uitvoer bekend is. Het doel van het algoritme is om een model te bouwen dat de relatie tussen invoer en uitvoer begrijpt en deze kan toepassen op nieuwe, ongeziene gegevens. Indien een waarde of getal voorspeld wordt, spreekt men van een regressie. Heeft de voorspelling betrekking op een groep of categorie, dan is er sprake van classificatie \autocite{Brownlee2023}. \\

Een voorbeeld van supervised learning is het voorspellen van huizenprijzen. Een model wordt daarbij getraind op een dataset van huizen waarvan de verkoopprijzen bekend zijn, samen met relevante kenmerken zoals locatie, grootte, aantal slaapkamers, enz. Het model leert vervolgens de relatie tussen deze kenmerken en de verkoopprijs van het huis. Als het model dan een nieuw huis met bijhorende kenmerken herkent, zal het de verkoopprijs ervan kunnen voorspellen.

\subsubsection{Unsupervised learning}

Unsupervised machine learning verloopt op een meer zelfstandige manier. Hierbij leert een algoritme om complexe processen en patronen te identificeren zonder de begeleiding van een mens. Bij unsupervised machine learning vindt training plaats op basis van data die geen labels of een specifieke, vooraf gedefinieerde output hebben  \autocite{Brownlee2023}. \\

Het misschien wel meest gekende voorbeeld van unsupervised learning zijn de aanbevelingen die streamingsplatformen zoals Netflix of Spotify aan hun kijkers aanbieden. Algoritmes analyseren het kijk- en luistergedrag van de gebruikers om daarin patronen en voorkeuren te ontdekken. Vervolgens kunnen dan gepersonaliseerde aanbevelingen gedaan worden.

\subsubsection{Reinforcement learning}

Bij reinforcement leert een algortime of cumputersysteem hoe het moet handelen in een omgeving om een bepaald doel te bereiken. Dit gebeurt door het uitvoeren van acties en het ontvangen van feedback in de vorm van beloningen of straffen. Wat reinforcement learning specifiek maakt is dat het leerproces vergelijkbaar is met hoe een mens of dier zou leren door trial-and-error en beloningen \autocite{Efimov2024}. \\

Reinforcement learning wordt gebruikt in robotica, bijvoorbeeld om robots autonoom te laten navigeren in een omgeving. Om een robot obstakels te leren vermijden wordt gewerkt met beloningen en straffen. Voor elke correcte beweging richting het doel zal de robot een beloning ontvangen. Omgekeerd zal de robot bestraft worden voor elke incorrecte beweging die gemaakt wordt. \\

De applicatie die als proof of concept ontwikkeld wordt voor deze bachelorproef, zal de toekomstige stroomproductie van zonnepanelen gaan voorspellen. Hiervoor zullen historische zonnestralingsgegevens gebruikt worden. Er zal dus gebruik gemaakt worden van de supervised learning methode en meer specifiek zal het regressiemodel toegepast worden. De waarde die moet voorspeld worden is de hoeveelheid zonnestraling uitgedrukt in Watt/m².

\subsection{\IfLanguageName{dutch}{Supervised machine learning: regressie modellen}{Supervised machine learning: regression models}}%
\label{sec:supervised machine learning: regressie modellen}

\subsubsection{Decision Tree Regression (DTR)}

Het Decision Tree model is een regressiealgoritme dat een dataset die gebruikt wordt om een bepaalde waarde te voorspellen systematisch opdeelt in steeds kleinere homgene subgroepen op basis van de kenmerken van die dataset. Het ontwikkelt daarbij een beslissingsboom, waarbij de interne knooppunten de kenmerken van de dataset vertegenwoordigen, de takken de beslissingsregles en elke bladknoop het resultaat. In de beslissingsboom zijn er twee knooppunten, namelijk de beslissingsknooppunten en de bladknooppunten. Beslissingsknooppunten worden gebruikt om een beslissing te nemen en hebben meerdere takken, terwijl Bladknooppunten de uitkomst van die beslissingen zijn en geen verdere takken bevatten.
De beslissingen of de test worden uitgevoerd op basis van de kenmerken van de gegeven dataset. Het doel is om eenvoudige beslissingsregels te leren op basis van die specifieke kenmerken van de dataset om zo een model te ontwikkelen dat een bepaalde waarde voorspelt  \autocite{Balakumar2023}.  
\\
\begin{figure}[h!]
    \centering\includegraphics[scale=0.55]{Decision_Tree }
    \caption{\label{fig:Decision_Tree}Grafische voorstelling beslissingsboom.}
\end{figure} 
\\

Decision trees zijn makkelijk te interpreteren omdat de geleerde beslissingsregels makkelijk te begrijpen en visualiseren zijn. Bovendien kan het DTR-algoritme niet-lineaire en dus meer complexe relaties tussen de inputvariabelen en de voorspelde waarde ontdekken, wat het algoritme zeer bruikbaar maakt voor datasets die complexe patronen bevatten  \autocite{Viswa2023}. 

\subsubsection{Random Forest Regression (RFR)}

Het Random Forest regressie model bestaat uit een verzameling van zogenaamde decision trees of beslissingsbomen. Een beslissingsboom is op zijn beurt een schema dat is opgebouwd uit een opeenvolging van binaire beslissingsregels. Het is een grafische voorstelling van een probleemstelling waarin verschillende mogelijke alternatieven met gebeurtenissen worden weergegeven en uitgewerkt. Het RFR-model maakt meerdere beslissingbomen aan op basis van willekeurig gekozen subsets van de gebruikte dataset. Het model voegt vervolgens de uitkomsten van al deze beslissingbomen samen om een algemene voorspelling te doen voor ongekende datapunten. Daarbij wordt telkens het gemiddelde of gewogen gemiddelde van elke beslissingsboom genomen. Op deze manier kan het grotere datasets verwerken en complexere verbanden vastleggen dan individuele beslissingsbomen. Het geheel van de voorspellingen van de beslissingbomen zorgt voor een grotere accuraatheid dan de voorspelling van één enkele beslissingsboom. Over het algemeen kan gesteld worden dat hoe meer beslissingsbomen er in het RFR-model zitten, des te robuster het model zal zijn \autocite{Balakumar2023}. \\

\begin{figure}[h!]
    \centering\includegraphics[scale=0.5]{Random_Forest}
    \caption{\label{fig:Random_Forest}Grafische voorstellingvan het Random Forest algoritme.}
\end{figure} 

Het RFR-model wordt vaak gebruikt om continue waarden te voorspellen, zoals aandelenkoersen, tijdreeksen of verkoopprijzen. Het is minder gevoelig voor 'overfitting' (dit doet zich voor wanneer een ML-model de trainingsdata te goed leert, waardoor het nieuwe ongekende data slecht voorspelt) dan andere regressiemethoden omdat het meerdere willekeurige bomen onafhankelijk van elkaar opbouwt en een gemiddelde van de individuele voorspellingen neemt. Het is een goede keuze wanneer data gebruikt wordt met veel verschillende kenmerken of inputvariabelen  \autocite{Sahai2023}.

\subsubsection{Support Vector Machine (SVM)}

Het Support Vector Machine algoritme probeert data zo optimaal mogelijk in twee groepen te verdelen door het vinden van het beste hyperplane. Deze hyperplane is de meest optimale scheidingslijn tussen de twee groepen van data. Het algoritme maakt daarbij gebruik van support vectors. Dit zijn de datapunten die zich het dichtst bij de optimale scheidingslijn bevinden. Om de meest optimale scheidingslijn te achterhalen, berekent het algoritme de maximale marge of afstand tussen de datapunten van de twee groepen  \autocite{Tziolis2024}. Omdat het algoritme zeer effectief is in het oplossen van complexere, niet-lineaire problemen, is het een model dat vaak grbuikt wordt om voorspellingen te maken op het gebied van hernieuwbare energie \autocite{Ahmad2018}. \\

\begin{figure}[h!]
    \centering\includegraphics[scale=0.6]{SVM}
    \caption{\label{fig:SVM}Grafische voorstelling Support Vector Machine.}
\end{figure} 

Het SVM-model wordt typisch gebruikt voor kleinere datasets omdat de trainigstijd van dit algoritme zeer lang kan zijn. Het wordt ook vooral toegepast op 'proper' datasets, dit zijn datasets die weinig afwijkingen of fouten bevatten.

\subsubsection{Autoregressive Integrated Moving Average (ARIMA)}

Het Autoregressive Integrated Moving Average (ARIMA) model staat voor een tijdreeksmodel dat gegevens uit het verleden gebruikt om de gegevens te interpreteren en voorspellingen te doen. Het ARIMA-model maakt gebruik van lineaire regressie. Het is een populaire en veelvuldig toegepaste statistische methode voor tijdreeksanalyses. Het is bijvoorbeeld te gebruiken voor het voorspellen van prijzen op basis van historische verdiensten. Univariate modellen zoals het ARIMA model zijn nuttig om de temperatuur te voorspellen. Het is ook mogelijk om seizoenspatronen mee te nemen in de voorspelling, maar dat komt in een volgend bericht aan de orde. Het ARIMA model wordt beschreven met drie parameters met dank aan de beschrijving van livingeconomyadvisors.

De term ‘AR’ in ARIMA staat voor autoregressie, wat aangeeft dat het model de afhankelijke relatie gebruikt tussen huidige gegevens en eerdere waarden. Met andere woorden, het laat zien dat de gegevens worden teruggebracht op de waarden uit het verleden.
De term ‘I’ staat voor geïntegreerd, wat betekent dat de gegevens stationair zijn. Stationaire gegevens verwijzen naar tijdreeksgegevens die “stationair” zijn gemaakt door de waarnemingen af ​​te trekken van de vorige waarden.
De term ‘MA’ staat voor moving average model, wat aangeeft dat de voorspelling of uitkomst van het model lineair afhangt van de waarden uit het verleden. Het betekent ook dat de fouten in prognoses lineaire functies zijn van fouten uit het verleden.

\subsubsection{Long Short-Term Memory (LSTM)}

Bidirectioneel Long Short-Term Memory (LSTM) is een type terugkerend neuraal netwerk (RNN) waarin de gegevens in twee richtingen worden verwerkt. Het is een speciaal type kunstmatig neuraal netwerk dat wordt gebruikt om temporele sequenties te modelleren. De bidirectionele LSTM biedt een krachtig hulpmiddel om temporele patronen vast te leggen en te herkennen, en helpt bij het nemen van efficiënte beslissingen bij besluitvormingsproblemen.

Bidirectionele LSTM's hebben een laag speciale geheugeneenheden die verbonden zijn met zowel activeringen van de voorgaande cellen als de huidige celactiveringen. Hierdoor heeft het model toegang tot verborgen toestandsinformatie uit zowel het verleden als de toekomst, waardoor het voorspellingen kan doen over de toekomstige toestanden op basis van de verborgen toestanden van cellen uit het verleden.

Het belangrijkste voordeel van de bidirectionele LSTM is de mogelijkheid om informatie in beide richtingen vast te leggen. Dit helpt het model bij het analyseren van afhankelijkheden over lange afstanden, en biedt generalisatie voor generalisatie naar onzichtbare gegevens. Bovendien kan de bidirectionele LSTM worden gebruikt om meerdere modellen voor verschillende taken te trainen, waardoor robuustere modellen kunnen worden gemaakt die complexere gegevens kunnen verwerken.

Bidirectionele LSTM's zijn gebruikt om een breed scala aan problemen te modelleren, variërend van natuurlijke taalverwerking, spraakherkenning, geluidsclassificatie en tijdree
ksvoorspellingen. Bovendien heeft het gebruik van bidirectionele LSTM om handgeschreven getallen uit de MNIST-dataset te herkennen state-of-the-art resultaten opgeleverd. De toepassing van bidirectionele LSTM's is ook onderzocht in verschillende andere domeinen, zoals emotieherkenning, automatische vertaling en beeldherkenning.

Bidirectionele LSTM's zijn een krachtige techniek om temporele patronen vast te leggen en zijn goed te generaliseren voor onzichtbare gegevens. 

\begin{figure}[h!]
    \centering\includegraphics[scale=0.5]{LSTM}
    \caption{\label{fig:LSTM}Grafische voorstellingvan het LSTM algoritme.}
\end{figure} 

\subsubsection{Extreme gradient Boosting (XGBoost)}

Het XGBoost algoritme is een ensemble-leermethode die gradiëntversterking en beslissingsbomen combineert. Het concept van gradiëntversterking verwijst naar het opeenvolgend toevoegen van beslissingsbomen aan het model, waarbij elke volgende boom de fouten corrigeert die door de vorige zijn gemaakt. Dankzij dit iteratieve proces kan het XGBoost zijn voorspellingen voortdurend verbeteren en een hoge nauwkeurigheid bereiken.De werking van beslissingsbomen is hierboven reeds toegelicht. \\

\begin{figure}[h!]
    \centering\includegraphics[scale=1.9]{XGBoost}
    \caption{\label{fig:XGBoost}Grafische voorstelling van de structuur van het XGBoost algoritme.}
\end{figure} 

Het XGBoost-model is specifiek ontworpen om de eigen prestaties te optimaliseren en grootschalige datasets efficiënt te verwerken. Door iteratief zwakke voorspellingen toe te voegen, kan het XGBoost algoritme snelle en accurate voorspellingen maken. Het grote voordeel van XGBoost is dat het ingebouwde mechanismen heeft om ontbrekende waarden in de dataset te verwerken. Het kan tijdens het trainingsproces automatisch leren hoe er het beste met ontbrekende waarden kan worden omgegaan. Vermits de meeste datasets uit de echte wereld gegevens ontbreken, is het XGBoost algoritme dus zeer geschikt om voorspellingen te maken. Nog een voordeel van het XGBoost-model is dat het makkelijk schaalbaar is, omdat het gebruik maakt van parallelle verwerkingstechnieken waarbij de werklast over meerdere cores of machines verdeeld wordt. Dit maakt snellere trainings- en voorspellingstijden mogelijk. Deze schaalbaarheid maakt XGBoost geschikt voor toepassingen die realtime voorspellingen vereisen.

\section{\IfLanguageName{dutch}{Elektriciteitsproductie voorspellen}{Predict electricity production}}%
\label{sec:elektriciteitsproductie voorspellen}

Doordat het belang van hernieuwbare energie de laatste jaren enorm is toegenomen, is er al heel wat onderzoek verricht naar het voorspellen van de toekomstige stroomproductie van zonnepanelen met behulp van machine learning. In de meeste van deze onderzoeken wordt deze voorspelling gebaseerd op de voorspelling van de hoeveelheid zonnestraling. Een andere optie bestaat erin om de historische stroomproductie van de zonnepanelen te gaan analyseren en daaruit een voorspelling te gaan maken \autocite{Wang2022}. Deze historische stroomproductie data is evenwel niet altijd voldoende voorhanden, zodat het soms niet mogelijk is om op bais daarvan voorspellinge te gaan doen.

\subsection{Zonnestraling}

Vermits zonnepanelen zonne-energie omzetten in elektricteit, is het bijna een logische keuze om de toekomstige stroomproductie van zonnepanelen te gaan voorspellen op basis van de voorspelling van de hoeveelheid zonnestraling \autocite{Ledmaoui2023}. Meestal wordt daarbij de inkomende of globale horizontale instraling (GHI, Global Horizontal Irradiance) voorspeld . GHI is de totale hoeveelheid zonnestraling die het aardoppervlak bereikt en wordt uitgedrukt in W/m². De GHI bestaat uit directe normale instraling (DNI, Direct Normal Irradiance)  en de diffuse horizontale instraling (DHI, Diffuse Horizontal Irradiance). DNI verwijst naar de grootste directe (90 graden) neerwaartse zonnestraling voor een bepaalde plaats. Deze directe normale instraling wordt gebruikt om de diffuse horizontale instraling te berekenen. De DHI is de zonnestraling die niet rechstreeks van de zon komt, maar verspreid is door de wolken en deeltjes in de atmosfeer. Deze zonnestraling komt in gelijke mate uit alle richtingen. \autocite{Sehrawat2023}. \\

Historische data met betrekking tot zonnestraling kan vrij verkregen worden via de CAMS Radiation Service (CRS) van de Copernicus Atmosphere Monitoring Service (\href{https://atmosphere.copernicus.eu}{CAMS}). Dit maakt deel uit van het European Earth observation programme Copernicus (EEC) dat informatiediensten biedt op basis van aardobservatiegegevens van satellieten en in-situgegevens (niet uit de ruimte). Enorme hoeveelheden wereldwijde gegevens van satellieten en van meetsystemen op de grond, in de lucht en op zee worden gebruikt om informatie te verstrekken aan dienstverleners, overheidsinstanties, andere internationale organisaties en burgers. De aangeboden informatiediensten zijn gratis en open toegankelijk voor de gebruikers ervan.

\subsection{Meteorologische data}

De hoeveelhied stroom die zonnepanelen produceren is niet enkel afhankelijk van de hoeveelheid zonnestraling. Ook andere meteorologische gegevens spelen hierbij een rol. Zo zijn de temperatuur, de relatieve luchtvochtigheid en de bewolkingsgraad factoren die het meeste invloed hebben op de stroomproductie van zonnepanelen \autocite{Sehrawat2023}. \\

Er zijn verschillende kanalen waarlangs historische weerdata kan verkregen worden. Zo zijn er datasets van meteorologische gegevens die door het Koninklijk Meteorologisch Insituuut (KMI) beschibaar gesteld worden. Het aantal datasets is echter zeer beperkt en moeten handmatig gedownload worden. Ook via het CAMS kan historische weerdata opgevraagd worden, maar ook deze data is beperkt aangezien de meest recente weerdata van het jaar 2016 is. \\

Er bestaan tal van open source en dus gratis weer API's die makkelijk geïntegreerd kunnen worden met een app. Deze API's bieden voorspellingen aan van verschillend meteorologische gegevens, maar leveren vaak beperkte historische weerdata aan. 
De open source \href{https://open-meteo.com/}{Open-Meteo API} levert historische weerdata tot 80 jaar terug in het verleden en biedt betrouwbare weersvoorspellingen aan tot 16 dagen in de toekomst. Deze weer API blijkt dus uitermate geschikt om meteorologische gegevens te verzamelen die samen met de zonnestralingsdata als dataset kunnen dienen om machine learning op toe te passen.


\bigskip
\bigskip
\bigskip
\bigskip
\bigskip
\bigskip
\bigskip
\bigskip
\bigskip
\bigskip
\bigskip
\bigskip
\bigskip
\bigskip
\bigskip
\bigskip
\bigskip
\bigskip
\bigskip
\bigskip
\bigskip
\bigskip
\bigskip
\bigskip
\bigskip
\bigskip
\bigskip
\bigskip
\bigskip
\bigskip


Omdat het noodzakelijk is dat gezinnen en bedrijven bewuster en actiever met hun energieverbuik gaan omspringen, wordt binnen Fluvius en ook andere bedrijven gekeken naar nieuwe technologieën en technieken \autocite{Verdoodt2018}. Dit bracht mij op het idee om een app te gaan ontwikkelen die wel actief en dus automatisch het elektriciteitsverbruik kan gaan bijsturen. Omdat er reeds apps bestaan die weerdata gebruiken om het elektriciteitsverbruik onder bepaalde weersomstandigheden in kaart te brengen, lijkt het me zeker mogelijk om weerdata ook als aansturing te gaan gebruiken, samen met de historieken van het elektriciteitsverbruik om dan zo het toekomstig elektriciteitsverbruik en de elektriciteitsproductie te voorspellen \autocite{Guo2022}. Er zijn tal van bestaande gratis weer API's, zoals \href{https://openweathermap.org/api}{Open Weather Map} of \href{https://www.weatherapi.com/}{Weather API} die kunnen geïntegreerd worden in een app. Om de voorspellingen zo accuraat mogelijk te maken, zal gebruik gemaakt worden van het Extreme Gradient Boosting (XGBOOST) algoritme \autocite{Ledmaoui2023}, \autocite{Wang2022} en \autocite{BarreraAnimas2022}. De historische verbruiksdata, de productiedata van de omvormer van de zonnepanelen en de weerdata zullen als input gebruikt worden voor dit algoritme. De data van de digitale elekticiteitsmeter en de omvormer van de zonnepanelen wordt continue uitgelezen met een python script op een Raspberry Pi computer die via het wifi-netwerk verbonden is. Deze data wordt opgelsagen in een NoSQL-databank die speciaal voor time-series ontwikkeld is, nl. InfluxDB \autocite{Balis2017} en  \autocite{Struckov2019}. De app zal tenslotte ontwikkeld worden voor iOS en gebouwd worden met Xcode, SwiftUI en UIKit \autocite{Allardice} en \autocite{Firtman2022}.

De eerste slimme toestellen die via de app zullen worden beheerd zijn zonnepanelen en een warmtepomp \autocite{Uytterhoeven2019}. Door de afschaffing van de virtueel terugdraaiende teller voor eigenaars van zonnepanelen, waarbij de teller van de elektriciteitsmeter terugdraait wanneer meer elektriciteit wordt opgewekt dan verbruikt, kan het verlies van dit voordeel opgevangen door het verbruik van de warmtepomp te laten samenvallen met de productiemomenten van de zonnepanelen \autocite{Selleslagh2021}. Wanneer echter uit de weersverwachtingen blijkt dat er een aanzienlijke elektriciteitsproductie zal zijn, zal de app  mede op basis van historiek van het elektriciteitsverbruik automatisch ook andere toestellen, zoals de vaatwas of wasmachine gaan inschakelen. Om de sanitaire toestellen te kunnen gaan inschakelen, zal in de eerste plaats gekeken worden of deze toestellen van zichzelf reeds slim zijn. Meer concreet zal geverifieerd worden of ze over een Soft Real Time Operating System beschikken (Soft RTOS) en via het wifinetwerk kunnen communiceren met een app. Voor de sanitaire toestellen die niet slim zijn, zal gebruik gemaakt worden van slimme stopcontacten. Daarbij wordt er tussen het klassieke stopcontact en de stekker van het toestel een apparaat geplaatst, waardoor een toestel op een eenvoudige manier slim kan gemaakt worden \autocite{Jong2020}.
